% main.tex - Orientation Change Detection Research (Draft)
\documentclass{article} % Replace with adonis "paper" if template installed
\usepackage[utf8]{inputenc}
\usepackage{graphicx}
\usepackage{amsmath}
\usepackage{siunitx}

\title{Lightweight Detection of Abrupt Orientation Changes Using a 6-Axis IMU}
\author{Your Name \\ Company/Institution}
\date{\today}

\begin{document}
\maketitle

\begin{abstract}
This work investigates simple yet robust algorithms for long-term monitoring of mast orientation using the ST ISM330DHCX inertial measurement unit. We focus on KISS-principle solutions that (i) track azimuth and altitude, (ii) ignore slow drift, and (iii) reliably detect abrupt orientation changes under outdoor noise.
\end{abstract}

\section{Introduction}
Brief context of mast monitoring, device configuration (WDS axis), and problem statement.

\section{Related Work}
Short survey of complementary filters, Kalman filtering, threshold-based detectors, adaptive drift compensation.

\section{Methodology}
\subsection{Sensor Model and Simulation}
Describe generated synthetic dataset (accelerometer $[-40,-120,-800]\,\text{mg}\pm50$; gyro $[12,-15,8]\,\text{dps}\pm20$) with an abrupt event.

\subsection{Algorithms Evaluated}
1. Tilt-only detection via low-pass accelerometer.
2. Complementary filter (acc + gyro) with pitch/roll correction.
3. Angular-rate spike detector.
4. Hybrid approach with validation timer.

\subsection{Evaluation Metrics}
Detection latency, false positives, energy cost.

\section{Results}
Summarise performance table and figures (to be filled once notebook evaluation complete).

\section{Discussion}
Analysis of trade-offs, recommended threshold selection, calibration procedure, long-term stability.

\section{Conclusion}
Key findings and future work.

\bibliographystyle{ieeetr}
\bibliography{references}

\end{document}