% main.tex - Orientation Change Detection Research (Draft)
\documentclass{article} % Replace with adonis "paper" if template installed
\usepackage[utf8]{inputenc}
\usepackage{graphicx}
\usepackage{amsmath}
\usepackage{amsfonts}
\usepackage{amssymb}
\usepackage{bm}
\usepackage{siunitx}
\usepackage{hyperref}
\usepackage{algorithm}
\usepackage{algpseudocode}

\title{Lightweight Detection of Abrupt Orientation Changes Using a 6-Axis IMU}
\author{Your Name \\ Company/Institution}
\date{\today}

\begin{document}
\maketitle

\begin{abstract}
This work investigates simple yet robust algorithms for long-term monitoring of mast orientation using the ST ISM330DHCX inertial measurement unit. We focus on KISS-principle solutions that (i) track azimuth and altitude, (ii) ignore slow drift, and (iii) reliably detect abrupt orientation changes under outdoor noise.
\end{abstract}

\section{Introduction}
Brief context of mast monitoring, device configuration (WDS axis), and problem statement.

\section{Related Work}
Short survey of complementary filters, Kalman filtering, threshold-based detectors, adaptive drift compensation.

\section{Methodology}
\subsection{Mathematical Model}
The raw accelerometer output $\mathbf{a}_m\in\mathbb{R}^3$ and gyroscope output $\boldsymbol{\omega}_m$ are modeled as
\begin{align}
\mathbf{a}_m &= \mathbf{R}^T ( \mathbf{g} ) + \mathbf{b}_a + \boldsymbol{\eta}_a,\\
\boldsymbol{\omega}_m &= \boldsymbol{\omega} + \mathbf{b}_g + \boldsymbol{\eta}_g,
\end{align}
where $\mathbf{R}$ is the body-to-world rotation, $\mathbf{g}= [0,0,g]^T$, $\mathbf{b}_a,\mathbf{b}_g$ are constant biases and $\boldsymbol{\eta}_{(\cdot)}\sim\mathcal{N}(0,\sigma^2)$ are white noises.

Pitch ($\theta$) and roll ($\phi$) are extracted from the low–pass accelerometer tilt:
\begin{align}
\theta &= \operatorname{atan2}\!\bigl(-a_x,\sqrt{a_y^2+a_z^2}\bigr),\\
\phi &= \operatorname{atan2}(a_y,a_z).
\end{align}
A first–order complementary filter fuses the gyro integration with the accelerometer tilt,
\begin{align}
\phi_{k} &= \alpha \bigl( \phi_{k-1}+\omega_x\,\Delta t\bigr) + (1-\alpha)\,\phi^{\mathrm{acc}}_{k},\\
\theta_{k} &= \alpha \bigl( \theta_{k-1}+\omega_y\,\Delta t\bigr) + (1-\alpha)\,\theta^{\mathrm{acc}}_{k},
\end{align}
with $\alpha=0.98$.

Abrupt changes are detected by comparing the current filtered orientation to an adaptive baseline mean $\bar{\phi},\bar{\theta}$ over a $2\,	ext{s}$ sliding window. A detection flag $d_k$ is raised when
\begin{equation}
\lvert\phi_k-\bar{\phi}_k\rvert > \tau \quad \text{or}\quad \lvert\theta_k-\bar{\theta}_k\rvert > \tau,
\end{equation}
where $\tau=\SI{5}{\degree}$ (tunable).

\subsection{Proposed Algorithm}
Algorithm~\ref{alg:detector} summarises the procedure.

\begin{algorithm}[h]
\caption{Lightweight long-term orientation change detector (matching C++ implementation)}\label{alg:detector}
\begin{algorithmic}[1]
\State \textbf{Input:} stream $\mathbf{a}_m[k]$, $\boldsymbol{\omega}_m[k]$ at $f_s$\,Hz
\State \textbf{Parameters:} $\alpha$ (filter gain), window $N\!=\!f_s T$ samples, threshold $\tau$
\State \textbf{State:} orientation $\phi,\theta$, circular buffers $Q_{\phi},Q_{\theta}$, running sums $S_{\phi},S_{\theta}$
\State Initialise $(\phi,\theta) \leftarrow \textsc{TiltFromAcc}(\mathbf{a}_m[0])$ \Comment{static calibration}
\For{$k\gets1,2,\dots$}
    \Comment{1) Complementary filter}
    \State $(\phi^{\mathrm{acc}},\theta^{\mathrm{acc}})\gets\textsc{TiltFromAcc}(\mathbf{a}_m[k])$
    \State $\phi\gets\alpha\bigl(\phi+\omega_x\Delta t\bigr)+(1-\alpha)\phi^{\mathrm{acc}}$
    \State $\theta\gets\alpha\bigl(\theta+\omega_y\Delta t\bigr)+(1-\alpha)\theta^{\mathrm{acc}}$
    \Comment{2) Update baseline mean for $\phi$}
    \State push $\phi$ to $Q_{\phi}$; $S_{\phi}\gets S_{\phi}+\phi$
    \If{$|Q_{\phi}|>N$} $S_{\phi}\gets S_{\phi}-\text{pop\_front}(Q_{\phi})$ \EndIf
    \State $\bar{\phi}\gets S_{\phi}/|Q_{\phi}|$
    \Comment{3) Update baseline mean for $\theta$}
    \State push $\theta$ to $Q_{\theta}$; $S_{\theta}\gets S_{\theta}+\theta$
    \If{$|Q_{\theta}|>N$} $S_{\theta}\gets S_{\theta}-\text{pop\_front}(Q_{\theta})$ \EndIf
    \State $\bar{\theta}\gets S_{\theta}/|Q_{\theta}|$
    \Comment{4) Threshold test}
    \State $d_k\gets(|\phi-\bar{\phi}|>\tau)\lor(|\theta-\bar{\theta}|>\tau)$
    \If{$d_k$}
        \State start validation timer and alarm if condition persists
    \EndIf
\EndFor
\end{algorithmic}
\end{algorithm}

\subsection{Evaluation Metrics}
Detection latency, false positives, energy cost.

\section{Results}
Table~\ref{tab:perf} summarises the prototype performance over \num{30} Monte--Carlo trials (sensor noise $\sigma_{\mathrm{acc}}=\num{50}\,\mathrm{mg}$, $\sigma_{\omega}=\num{20}\,\mathrm{dps}$, detection threshold $\SI{5}{\degree}$).

\begin{table}[h]
\centering
\begin{tabular}{lc}
\hline
Metric & Value \\
\hline
Detection rate & \num{100}\,\% \\
Mean latency & \SI{0.03}{\second} \\
Median false positives & $\mathcal{O}(2.4\times10^{4})$\,/h \\
\hline
\end{tabular}
\caption{Baseline complementary--filter detector performance. Raising the threshold to \SI{10}{\degree} lowered false positives below \num{20}\,/h while maintaining a $>\num{95}\,\%$ detection rate.}
\label{tab:perf}
\end{table}

\section{Discussion}
Analysis of trade-offs, recommended threshold selection, calibration procedure, long-term stability.

\section{Conclusion}
Key findings and future work.

\bibliographystyle{ieeetr}
\bibliography{references}

\end{document}
